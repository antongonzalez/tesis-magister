\chapter{Transformación inversa para los tensores de marea}
\label{ape:4}

Para demostrar las ecuaciones \eqref{eq:52} tenemos que a partir de la definición \eqref{eq:51} se puede construir que
\begin{align}
\epsilon_{\alpha \beta \mu \sigma} \mathtt{B}^{\mu}_{\ \gamma} u^{\sigma} &= \frac{1}{2} \epsilon_{\alpha \beta \mu \sigma} \epsilon^{\delta \lambda \rho \mu} \partial_{\gamma} F_{\rho \lambda} u_{\delta} u^{\sigma}\\
&= \partial_{\gamma} F_{\mu \nu} + 2u_{[\alpha} \mathtt{E_{\beta] \gamma}},
\end{align}
de donde se puede deducir que
\begin{equation}
\partial_{\gamma} F_{\mu \nu} =  \epsilon_{\alpha \beta \mu \sigma} \mathtt{B}^{\mu}_{\ \gamma} u^{\sigma} - 2u_{[\alpha} \mathtt{E_{\beta] \gamma}}.
\end{equation}

Para el caso gravitacional tenemos que, de forma similar, se puede construir que
\begin{align}
\epsilon_{\alpha \beta \mu \sigma} \mathbb{H}^{\mu}_{\ \gamma} u^{\sigma} &= \frac{1}{2} \epsilon_{\alpha \beta \mu \sigma} \epsilon^{\rho \lambda \mu \delta} R_{\rho \lambda \gamma \xi} u_{\delta} u^{\xi} u^{\sigma}\\
&= R_{\alpha \beta \xi \gamma} u^{\xi} + 2u_{[\alpha} \mathbb{E}_{\beta] \gamma},
\end{align}
de donde se deduce que
\begin{equation}
R_{\alpha \beta \gamma \xi} u^{\xi} = \epsilon_{\alpha \beta \mu \sigma} \mathbb{H}^{\mu}_{\ \gamma} u^{\sigma} - 2u_{[\alpha} \mathbb{E}_{\beta] \gamma}.
\end{equation}
