\chapter*{Resumen}

En la presente se estudia una analogía propuesta, basada en tensores de marea, entre la teoría electromagnética clásica y la teoría de Relatividad General. Para esto se presentan ambas teorías de una forma compacta a modo de introducción, luego se utiliza la expansión multipolar para obtener las ecuaciones de movimiento para cuerpos con estructura en Relatividad General y la teoría electromagnética clásica, para luego compararlas en los órdenes dipolar y cuadrupolar.

A modo de seguir comparando las dos teorías utilizando esta analogía, se analizan órbitas circulares en la métrica de Kerr para el caso de cuerpos extendidos hasta orden dipolar.