\chapter*{Resumen}

Estudiamos la expansión multipolar en relatividad general para dos casos particulares. En el primero estudiamos si es posible extender, a órdenes superiores, un análisis realizado por Costa \& Herdeiro \cite{Costa-Herdeiro} donde se muestra una analogía entre un giroscopio y un dipolo magnético. Esto permite obtener las ecuaciones de Mathisson-Papapetrou a partir de un caso electromagnético en relatividad especial, relacionando así la teoría de Relatividad General y la teoría electromagnética a través de los tensores de marea. Para esto se presenta la teoría electromagnética clásica y la teoría de Relatividad General de una forma compacta a modo de introducción y luego, utilizando la expansión multipolar, se obtienen las ecuaciones de movimiento para cuerpos con estructura interna tanto en Relatividad General como en la teoría electromagnética clásica en los órdenes dipolar y cuadrupolar. Finalmente, al extender el análisis realizado a orden dipolar en \cite{Costa-Herdeiro}, observamos que la cantidad de componentes linealmente independientes necesarias para describir de forma completa la dinámica de cuerpos extendidos en Relatividad General, a orden cuadrupolar,  es mayor que en electromagnetismo. Esto es producto de que la cantidad de componentes linealmente independientes de la curvatura (6+8+6) es mayor a la cantidad de componentes linealmente independientes necesarias para describir de forma completa el movimiento de un dipolo magnético en la teoría electromagética clásica (16 en total). A razón de esto se introduce un nuevo tensor de marea en el caso gravitacional, el cual no tiene análogo electromagnético. Esto provoca que no se mantenga la analogía a orden quadrupolar.

En el segundo estudio de la expansión multipolar se obtienen soluciones circulares a las ecuaciones de Mathisson-Papapetrou sobre el plano ecuatorial en la métrica de Kerr. Estas nos permiten observar la existencia de una ``velocidad angular crítica'' $\omega_{\mathrm{critical}}$, valor para el cual el espín del giróscopo y la norma del 4-momentum divergen a $-\infty$. Asimismo, observamos que para valores de velocidad angular cercanos a $\omega_{\mathrm{critical}}$ el 4-momentum es un vector tipo espacio. Esto nos lleva a pensar que existen zonas, en el espacio de velocidades angulares, para las que parece ser necesario considerar un mayor número de órdenes multipolares, o bien, restricciones adicionales sobre los parámetros de la órbita para una correcta descripción en la dinámica del giróscopo. Situaciones particulares también son estudiados al final de este trabajo como, por ejemplo, casos estacionarios, soluciones con cierta velocidad angular específica con las que se pueden obtener trayectorias geodésicas, casos en la métrica de Schwarzschild y Minkowski, donde además se generalizan los resultados obtenidos por Costa \& Herdeiro en su estudio de órbitas helicoidales.