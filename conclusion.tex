\chapter{Conclusiones y trabajos futuros}

En la presente tesis se ha estudiado la analogía gravitoelectromagnética propuesta por Costa \& Herdeiro. Hemos extendido su análisis con el fin de introducir órdenes superiores en la expansión multipolar. Comparamos de una forma general la teoría de Relatividad General y la teoría electromagnética, además de estudiar soluciones particulares a las ecuaciones de Mathisson-Papapetrou como las presentadas en el capítulo \ref{cap:4}.

La discusión realizada en \cite{Costa-Herdeiro}, que permite obtener las ecuaciones de Mathisson-Papapetrou a partir de un caso electromagnético, es solo válida para las ecuaciones que determinan la evolución del 4-momentum en el orden dipolar. Esto no ocurre con las ecuaciones que determinan la dinámica del espín, ni para órdenes superiores producto de que los efectos de inducción electromagnética son propios de la teoría electromagnética clásica. Adicionalmente, observamos que no es posible obtener las ecuaciones de movimiento en Relatividad General a partir de las ecuaciones análogas en la teoría electromagnética para distribuciones de masa/carga más generales, modeladas con los órdenes dipolar y quadrupolar. Esto ocurre producto de que la cantidad de elementos necesarios para describir la curvatura del espaciotiempo es mayor a la cantidad de elementos necesarios para describir las inhomogeneidades de los campos electromagnéticos, lo cual se ve reflejado en \eqref{eq:115} y \eqref{eq:espin-grav}. Así, al descomponer las 20 componentes linealmente independientes del tensor de curvatura en los tensores de marea $\{ \mathbb{E}, \mathbb{H}, \mathbb{F} \}$, observamos que la cantidad de tensores de marea necesarios para codificar toda la información de la curvatura, en Relatividad General, es mayor a la cantidad de tensores de marea necesarios para codificar la información de las inhomogeneidades de los campos electromagnéticos. Luego, al escribir las ecuaciones de movimiento a orden cuadrupolar en Relatividad General, éstas quedan escritas en términos de $\mathbb{F}_{\mu \nu}$, el cual no posee análogo electromagnético. No así en el caso dipolar, donde $\mathbb{F}_{\mu \nu}$ no está presente en las ecuaciones de movimiento. De esta forma concluimos que, para casos generales, la analogía existente a orden dipolar no puede ser extendida a órdenes superiores.

No hay precedentes en la literatura de que la discusión hecha para extender el análisis realizado por Costa \& Herdeiro, al comparar un giróscopo con un dipolo magnético, se hayan realizado previamente. Por lo que los resultados contenidos en es mostrados en esta tesis son nuevos y permiten un mejor entendimiento de las diferencias entre la teoría electromagnética clásica y Relatividad General.

De forma adicional, se han estudiado soluciones circulares bajo la condición suplementaria de Mathisson-Pirani, ver capítulo \ref{cap:4}. Se ha logrado analizar el comportamiento del 4-momentum del giróscopo con el fin de encontrar zonas en las cuales es un 4-vector tipo tiempo o tipo espacio. Y, como es de esperar, el comportamiento causal del 4-momentum depende de los distintos parámetros de la órbita en cuestión tales como, el radio $R$ de la órbita, la masa $M$ del giróscopo, las características del cuerpo central $a,\ m$ y su velocidad tangencial $v$. Esta última está relacionada con la velocidad angular $\omega$ del cuerpo. Además, existe un valor límite para la velocidad angular producto de que la velocidad tangencial del cuerpo no puede superar la velocidad de la luz. Es por esto que los valores de velocidades angulares permitidos para la órbita son los que se encuentran en el intervalo $]\omega_{\mathrm{min}}, \omega_{\mathrm{max}}[$. Si nos centramos en estudiar valores en ese rango, encontramos que para los espaciotiempos de Schwarzschild y Kerr, dentro de ese intervalo existe un valor crítico de velocidad angular para la cual el espín del giróscopo diverge a $-\infty$. Esto causa que el 4-momentum también lo haga, provocando así que en zonas cercanas a la divergencia este sea un vector tipo espacio, como se puede ver en la zona $]\omega_-, \omega_+[$ en las figuras \ref{fig:2} y \ref{fig:3}.

Casos particulares de la solución obtenida también son analizados, logrando reproducir los resultados presentados en \cite{Costa-Herdeiro-Natario-Zilhao} donde órbitas circulares bajo la condición de Mathisson-Pirani son estudiadas en un espaciotiempo de Minkowski. Destacamos además el caso para el que la velocidad angular del cuerpo es dada por $\omega = a/(a^2+R^2)$, ya que representa el ejemplo de una órbita circular donde el comportamiento causal del 4-momentum es un indicio que nos lleva a pensar que podría ser necesario considerar un mayor número de órdenes en al expansión multipolar para una correcta descripción del problema. Las soluciones discutidas aquí deberían ser útiles para comprender los límites de validez de la aproximación polo-dipolo. 

Es fundamental destacar que el estudio de las soluciones a las ecuaciones de Mathisson-Papapetrou es importante para comprender de forma más amplia la dinámica completa de cuerpos en Relatividad General. Y, al no haber estudios similares, bajo la condición suplementaria de Mathisson-Pirani, a los realizados en la presente tesis, enfatizamos la relevancia de los resultados mostrados en el capítulo \ref{cap:4}. Los que, a modo de apreciación personal, considero lo más relevante en mi trabajo.

A lo largo de este trabajo se han abierto algunos problemas los cuales pueden ser tratados en trabajos futuros, como lo son:
\begin{itemize}
\item Estudiar espaciotiempo particulares donde sea posible establecer la misma relación entre las ecuaciones de movimiento \eqref{eq:pele}-\eqref{eq:66}, y donde se pueda escribir el tensor de marea \eqref{eq:mag-mag} en términos del tensor gravito-magnético y gravito-electros de marea definidos en el capítulo \ref{cap:3}.
\item Realizar un análisis detallado para la estabilidad del órbitas estudiadas en el capítulo \ref{cap:4}. Es decir, a partir de la solución obtenida en la presente tesis, considerar una pequeña perturbación dependiente del tiempo y a partir de las ecuaciones de movimiento, determinar la evolución de la perturbación.
\item Estudiar cómo la inclusión de términos de orden superior, es decir, en el nivel de cuadrupolo y más allá, podría aliviar las patologías de las soluciones de las ecuaciones de movimiento de un dipolo gravitacional.
\end{itemize}