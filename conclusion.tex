\chapter{Conclusiones y trabajos futuros}

En la presente tesis se ha estudiado la analogía gravitoelectromagnética propuesta por Costa \& Herdeiro, en la cual se ha extendido su análisis con el fin de introducir órdenes superiores en la expansión multipolar para comparar de una forma más general ambas teorías, además de estudiar y resolver casos particulares como el presentado en el capítulo \ref{cap:4}, de esta forma las principales conclusiones son:
\begin{itemize}
\item La discusión realizada en \cite{Costa-Herdeiro} no puede ser extendida a orden cuadrupolar, esto es por que a dicho orden es necesaria la información completa del tensor de Riemann para determinar las ecuaciones de movimiento de los objetos en cuestión, y dicho tensor no puede ser reconstruido completamente utilizando los tensores de marea, teniendo así que la analogía propuesta basada en tensores de marea no permite una exacta correspondencia entre ambas teorías como sí ocurría en el caso dipolar.
\item A partir de la solución obtenida en el capítulo \ref{cap:4} se pueden deducir diversas características a satisfacer por parte de los satélites que realizan una órbita circular al rededor de un agujero negro rotante en el plano ecuatorial.
\end{itemize}

Sin embargo, a lo largo de este trabajo se han abierto algunos problemas los cuales pueden ser tratados en trabajos futuros, como lo son:
\begin{itemize}
\item Obtener una solución similar al problema resuelto en el capítulo \ref{cap:4} pero con una velocidad angular arbitraria. Esto entregaría una solución más general a dicho problema a costa de hacer no-nulo el lado derecho de las ecuaciones de movimiento en \eqref{eq:66}, puesto que permitiría aumentar los grados de libertad del sistema al no existir una relación previamente fijada entre las coordenadas $u^t$ y $u^{\phi}$.
\item Realizar un análisis detallado para la estabilidad de la órbita.
\end{itemize}