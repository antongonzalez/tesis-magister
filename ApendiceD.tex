\chapter{Orden menor en la expansi\'on para una part\'icula monopolo-dipolo}
\label{ape:3}

Para obtener \eqref{eq:43} usaremos el teorema \ref{teo:2} en \eqref{eq:37}, de donde se obtiene que del orden m\'as bajo
\begin{equation}
\label{eq:94}
\frac{\delta^2}{\mathrm{d}s^2} (t^{\gamma \rho \nu} u_{\gamma}u_{\rho}) + \cd{} (t^{\gamma \nu} u_{\gamma} - 2\cd{u_{\gamma}} u_{\rho} t^{(\gamma \rho) \nu}) + \frac{1}{2} R^{\ \ \ \ \nu}_{\mu \gamma \delta} (2 u^{\mu} u_{\rho} t^{\hat{\gamma} \rho \delta} + t^{\hat{\gamma} \hat{\mu} \delta}) = 0.
\end{equation}

Trabajando cada t\'ermino por separado se tiene que
\begin{align}
\label{eq:95}
\cd{} \left( t^{\gamma \sigma \nu} u_{\gamma} u_{\sigma} \right) &= \cd{u_{\sigma}} \des{1}{t}^{\sigma \nu} + u_{\gamma} \cd{\des{1}{t}^{\sigma \nu}},\\
\label{eq:96}
t^{\gamma \nu} u_{\gamma} &= \ \des{0}{o}^{\nu} + \des{0}{t}^{\nu} u^{\nu},\\
\label{eq:97}
\cd{u_{\gamma}} u_{\sigma} t^{\gamma \sigma \nu} &= \cd{u_{\gamma}} \left( \des{1}{o}^{\gamma \nu} + \des{1}{o}^{\gamma} u^{\nu} \right),\\
\label{eq:98}
\cd{u_{\gamma}} u_{\sigma} t^{\sigma \gamma \nu} &= \cd{u_{\gamma}} \des{1}{t}^{\gamma \nu}.\\
\label{eq:100}
2 t^{\hat{\gamma} \sigma \rho} u^{\mu} u_{\sigma} &= 2u^{\mu} \left( \des{1}{o}^{\gamma \rho} + \des{1}{o}^{\gamma} u^{\rho} \right),\\
\label{eq:101}
t^{\hat{\gamma} \hat{\mu} \rho} &= \ \des{1}{o}^{\gamma \rho \mu} + \des{1}{o}^{\gamma \mu} u^{\rho}
\end{align}

Reemplazando \eqref{eq:95}, \eqref{eq:96}, \eqref{eq:97} y \eqref{eq:98} en los primeros dos t\'erminos de \eqref{eq:94} se obtiene que
\begin{align}
& \frac{\delta^2}{\mathrm{d}s^2} (t^{\gamma \rho \nu} u_{\gamma}u_{\rho}) + \cd{} (t^{\gamma \nu} u_{\gamma} - 2\cd{u_{\gamma}} u_{\rho} t^{(\gamma \rho) \nu}) \nonumber\\
& \quad = \cd{} \left( \cd{} \left( t^{\gamma \sigma \nu} u_{\gamma \sigma} \right) + t^{\gamma \nu} u_{\gamma} - \cd{u_{\gamma}} u_{\sigma} t^{\gamma \sigma \nu} - \cd{u_{\gamma}} u_{\sigma} t^{\sigma \gamma \nu} \right) \nonumber \\
\label{eq:99}
& \quad = \cd{} \left( u_{\gamma} \cd{} \des{1}{t}^{\gamma \nu} + \des{0}{o}^{\nu} + \des{0}{t} u^{\nu} - \cd{u_{\gamma}} \des{1}{o}^{\gamma \nu} - \cd{\gamma} \des{1}{o}^{\gamma}u^{\nu} \right).
\end{align}

Y reemplazando \eqref{eq:100} y \eqref{eq:101} en el tercer t\'ermino de \eqref{eq:37} se obtiene que
\begin{equation}
\label{eq:102}
 \frac{1}{2} R^{\ \ \ \ \nu}_{\mu \gamma \delta} (2 u^{\mu} u_{\rho} t^{\hat{\gamma} \rho \delta} + t^{\hat{\gamma} \hat{\mu} \delta}) = \frac{1}{2} R^{\ \ \ \ \nu}_{\mu \gamma \rho} \left[ 2u^{\mu}(\des{1}{o}^{\gamma \rho} + \des{1}{o}^{\gamma} u^{\rho}) + \des{1}{o}^{\gamma \mu \rho} + \des{1}{o}^{\gamma \mu}  u^{\rho}  \right],
\end{equation}
luego, de \eqref{eq:99} y \eqref{eq:102} se obtiene \eqref{eq:43}.