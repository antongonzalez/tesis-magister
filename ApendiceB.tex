\chapter{Ecuaciones de movimiento para una partícula monopolo-dipolo}
\label{ape:2}

Para obtener \eqref{eq:82} trabajaremos el primer t\'ermino de \eqref{eq:43}, as\'i
\begin{align}
& \cd{} \left( u_{\gamma} \cd{} \des{1}{t}^{\gamma \nu} + \des{0}{o}^{\nu} + \des{0}{t} u^{\nu} - \cd{u_{\gamma}} \des{1}{o}^{\gamma \nu} - \cd{\gamma} \des{1}{o}^{\gamma}u^{\nu} \right) \nonumber\\
&\quad = \cd{} \left[ u_{\gamma} \des{1}{t}^{\gamma \nu} -  \rho^{\nu}_{\ \delta} u_{\gamma} \cd{} \left( \des{1}{o}^{\delta \gamma} + \des{1}{o}^{\delta} u^{\gamma} + \des{1}{t}^{\delta \gamma} \right) + \des{0}{t} u^{\nu} - \cd{u_{\gamma}} \des{1}{o}^{\gamma \nu} -  \cd{u_{\gamma}} \des{0}{o}^{\gamma} u^{\nu} \right] \nonumber \\
&\quad = \cd{}  \left[ u^{\nu} u_{\gamma} u_{\sigma} \cd{} \left( \des{1}{o}^{\gamma} u^{\sigma} \right) + u_{\gamma} u_{\sigma} u^{\nu} \cd{} \des{1}{t}^{\gamma \sigma} + \des{0}{t} u^{\nu} - u_{\gamma} \cd{\des{1}{o}^{\nu \gamma}} - u_{\gamma} \cd{\des{1}{o}^{\nu} u^{\gamma}} - \cd{u_{\gamma}} \des{1}{o}^{\gamma \nu} \right. \nonumber \\
&\qquad \left. - \cd{u_{\gamma}} \des{1}{o}^{\gamma} u^{\nu} \right] \nonumber \\
&\quad = \cd{}  \left[  -u^{\nu} u_{\gamma} \cd{u_{\sigma} \des{1}{S}^{\gamma \sigma}} + u_{\gamma} u_{\sigma} u^{\nu} \cd{\des{1}{t}^{\gamma \sigma}} + \des{0}{t} u^{\nu} + u_{\gamma} \left(  \frac{1}{2} \des{1}{S}^{\nu \gamma} + u_{\sigma} \des{1}{S}^{\sigma [\nu} u^{\gamma]} \right) \right. \nonumber \\
&\qquad \left. + \cd{} \left( u_{\gamma} \des{1}{S}^{\nu \gamma} \right) + \frac{1}{2} \cd{u_{\gamma}} \left( \des{1}{S}^{\gamma \nu} + 2u_{\sigma} \des{1}{S}^{\sigma [\gamma} u^{\nu]} \right) + \cd{u_{\gamma}} u_{\sigma} \des{1}{S}^{\gamma \sigma} u^{\nu} \right] \nonumber\\
&\quad = \cd{} \left[ u^{\nu} u_{\gamma} \cd{u_{\sigma}} \des{1}{S}^{\sigma \gamma} + u_{\gamma} u_{\sigma} \cd{\des{1}{t}^{\gamma \sigma}} u^{\nu} + \des{0}{t} u^{\nu} + \frac{1}{2} \cd{} \left(       u_{\gamma} \des{1}{S}^{\gamma \nu} \right) - \frac{1}{2} u_{\sigma} \cd{} \left( u_{\gamma} u^{\nu} \des{1}{S}^{\gamma \sigma} \right) \right. \nonumber \\
&\qquad \left. + \cd{}\left( u_{\gamma} \des{1}{S}^{\nu \gamma} \right) + \frac{1}{2} u_{\gamma} \cd{\des{1}{S}^{\nu \gamma}} + \frac{1}{2} \des{1}{S}^{\gamma \nu} \cd{u_{\gamma}} + \frac{1}{2} \cd{u_{\gamma}} u_{\sigma} \des{1}{S}^{\gamma \sigma} u^{\nu} \right] \nonumber
\end{align}
\newpage

\begin{align}
&\quad = \cd{} \left[ u^{\nu} u_{\gamma} \cd{u_{\sigma}} \des{1}{S}^{\sigma \gamma} + u_{\gamma} u_{\sigma} \cd{\des{1}{t}^{\gamma \sigma}} u^{\nu} + \des{0}{t} u^{\nu} - \frac{1}{2} \cd{} \left(       u_{\gamma} \des{1}{S}^{\nu \gamma} \right) - \frac{1}{2} u_{\sigma} \cd{} \left( u_{\gamma} u^{\nu} \des{1}{S}^{\gamma \sigma} \right) \right. \nonumber \\
&\qquad \left. + \cd{}\left( u_{\gamma} \des{1}{S}^{\nu \gamma} \right) + \frac{1}{2} u_{\gamma} \cd{\des{1}{S}^{\nu \gamma}} - \frac{1}{2} \des{1}{S}^{\nu \gamma} \cd{u_{\gamma}} + \frac{1}{2} \cd{u_{\gamma}} u_{\sigma} \des{1}{S}^{\gamma \sigma} u^{\nu} \right] \nonumber\\
&\quad = \cd{} \left[ u^{\nu} u_{\gamma} \cd{u_{\sigma}} \des{1}{S}^{\sigma \gamma} + u_{\gamma} u_{\sigma} \cd{\des{1}{t}^{\gamma \sigma}} u^{\nu} + \des{0}{t} u^{\nu} + \frac{1}{2} \cd{} \left(       u_{\gamma} \des{1}{S}^{\nu \gamma} \right) - \frac{1}{2} u_{\sigma} \cd{} \left( u_{\gamma} u^{\nu} \des{1}{S}^{\gamma \sigma} \right) \right. \nonumber \\
&\qquad \left. +\frac{1}{2} u_{\gamma} \cd{\des{1}{S}^{\nu \gamma}} - \frac{1}{2} \des{1}{S}^{\nu \gamma} \cd{u_{\gamma}} + \frac{1}{2} \cd{u_{\gamma}} u_{\sigma} \des{1}{S}^{\gamma \sigma} u^{\nu} \right] \nonumber\\
& \qquad= \cd{} \left[ \left( \des{0}{t} - u_{\gamma} \cd{u_{\rho}} \des{1}{S}^{\gamma \rho} + u_{\gamma} u_{\rho} \cd{\des{1}{t}^{\gamma \rho}} \right) u^{\mu} + u_{\gamma} \cd{} \des{1}{S}^{\mu \gamma} \right] \nonumber\\
\label{eq:92}
&\quad = \cd{p^{\nu}}.
\end{align}

Por otro lado, si trabajamos el segundo t\'ermino en \eqref{eq:43} se tiene que
\begin{align}
& \frac{1}{2} R^{\ \ \ \ \nu}_{\mu \gamma \rho} \left[ 2u^{\mu}(\des{1}{o}^{\gamma \rho} + \des{1}{o}^{\gamma} u^{\rho}) + \des{1}{o}^{\gamma \mu \rho} + \des{1}{o}^{\gamma \mu}  u^{\rho}  \right] \nonumber \\
& \quad = \frac{1}{2} R^{\ \ \ \ \nu}_{\mu \gamma \rho} \left[ 2 u^{\mu} \left( -\frac{1}{2} \des{1}{S}^{\gamma \rho} - \frac{1}{2} u_{\sigma} \des{1}{S}^{\sigma \gamma} u^{\rho} + \frac{1}{2} u_{\sigma} \des{1}{S}^{\sigma \rho} u^{\gamma} - u_{\sigma} \des{1}{S}^{\gamma \sigma} u^{\rho} \right) \right. \nonumber \\
&\qquad \left. - \frac{1}{2} \des{1}{S}^{\gamma \mu} u^{\rho} - \frac{1}{2} u_{\sigma} \des{1}{S}^{\sigma \gamma} u^{\mu} u^{\rho} + \frac{1}{2} u_{\sigma} \des{1}{S}^{\sigma \mu} u^{\gamma} u^{\rho}  \right] \nonumber\\
& \quad = - \frac{1}{2} R^{\ \ \ \ \nu}_{\mu \gamma \rho} \des{1}{S}^{\gamma \rho} u^{\mu} - \frac{1}{4} R^{\ \ \ \ \nu}_{\mu \gamma \rho} \des{1}{S}^{\gamma \mu} u^{\rho}, \nonumber
\end{align}
y usando que $R_{[\mu \nu \gamma]}^{\ \ \ \ \sigma} = 0$, se obtiene que
\begin{equation}
\label{eq:93}
\frac{1}{2} R^{\ \ \ \ \nu}_{\mu \gamma \rho} \left[ 2u^{\mu}(\des{1}{o}^{\gamma \rho} + \des{1}{o}^{\gamma} u^{\rho}) + \des{1}{o}^{\gamma \mu \rho} + \des{1}{o}^{\gamma \mu}  u^{\rho} \right] = \frac{1}{2} R^{\ \ \ \ \nu}_{\gamma \mu \rho} \des{1}{S}^{\gamma \mu} u^{\rho}.
\end{equation}

De esta forma, usando \eqref{eq:92} y \eqref{eq:93} se obtiene directamente \eqref{eq:82}.


