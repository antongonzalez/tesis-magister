\chapter*{Introducci\'on}

En la presente se estudia una analogía propuesta, basada en tensores de marea, entre la teoría electromagnética clásica y la teoría de Relatividad General. En \cite{Costa-Herdeiro} se plantea una nueva forma de establecer una relación entre la teoría de Relatividad General y la teoría electromagnética clásica. Los tensores de marea, que permiten el estudio de las inhomogeneidades de los campos en cuestión, al ser usados para escribir las ecuaciones de movimiento en ambas teorías permiten dar cuenta de una completa correspondencia entre los elementos que describen el movimiento de un dipolo magnético y una partícula con espín \cite{Steinhoff-Puetzfeld}.

Para poder entender lo anterior de una forma más clara, en el capítulo \ref{cap:1} se introducen temas relacionados con la teoría electromagnética clásica \cite{gr2}, desde las ecuaciones de Maxwell (y su respectiva formulación covariante), la obtención de la ecuación de desvío para un par de cargas de prueba, finalizando con la expansión multipolar \cite{Dixon1,Dixon2,Dixon3} y el estudio particular para un dipolo magnético.

Luego, en el capítulo \ref{cap:2} se introducen los distintos elementos de Relatividad General \cite{gr1,inverno,Steven,Hans,Poisson}, desde presentar las ecuaciones de campo de Einstein, explicando el principio de equivalencia y describiendo la solución de Kerr \cite{Heinicke}, para después obtener la ecuación de desvío geodésico, y finalizando con la expansión multipolar gravitacional hasta orden cuadrupolar.

Resumiendo ambas teorías en los capítulos anteriores, en el capítulo \ref{cap:3} se presenta en detalle la analogía propuesta en \cite{Costa-Herdeiro}, para luego hacer uso de las ecuaciones de desvío obtenidas anteriormente para definir los tensores de marea electromagnéticos y gravitacionales. Después, se hace una primera comparación para el caso de campos gravitacionales débiles. Y para finalizar, se estudian extensiones a órdenes superiores a la correspondencia existente para el caso dipolar en las ecuaciones de movimiento. Finalizando con una discusión sobre las similitudes y diferencias entre la teoría de Relatividad General y la teoría electromagnética clásica.

\newpage
Por último, en el capítulo \ref{cap:4} se estudia un caso particular en donde al considerar una 4-velocidad de la forma $u^{\mu}=\left[ u^t, 0, 0, u^{\phi} \right]$, se simplifican las ecuaciones de movimiento al punto de que es posible obtener soluciones analíticas. Este análisis surge a modo de complementar una discusión presentada en \cite{Costa-Natario-Zilhao}, en donde se comparan soluciones circulares en la métrica de Kerr y fuera de una distribución esférica de carga con momento angular $J$, para partículas de prueba dipolares.