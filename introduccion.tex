\chapter*{Introducci\'on}

En la presente se estudia una analogía propuesta, basada en tensores de marea, entre la teoría electromagnética clásica y la teoría de Relatividad General. En \cite{Costa-Herdeiro} se plantea una nueva forma de establecer una relación entre la teoría de Relatividad General  y la teoría electromagnética clásica. A partir de los tensores de marea, los cuales permiten el estudio de las inhomogeneidades de los campos en cuestión, es se logra obtener una completa correspondencia entre los términos que describen el movimiento de un dipolo magnético y una partícula con espín \cite{Steinhoff-Puetzfeld}.

Para poder entender lo anterior de una forma más clara, en el capítulo \ref{cap:1} se introducen los temas relacionados con la teoría electromagnética clásica \cite{gr2}, desde las ecuaciones de Maxwell (y su respectiva formulación covariante), la obtención de la ecuación de desvío para un par de carga de prueba, finalizando con la expansión multipolar \cite{Dixon1,Dixon2,Dixon3} y el estudio del caso particular para un dipolo magnético.

Luego, en el capítulo \ref{cap:2} se introducen los distintos elementos de Relatividad General \cite{gr1,inverno,Steven,Hans,Poisson} desde presentar las ecuaciones de campo de Einstein, explicando el principio de equivalencia y describiendo la solución de Kerr \cite{Heinicke}, para después obtener la ecuación de desvío geodésico, y finalizando con la expansión multipolar gravitacional hasta orden cuadrupolar.

Habiendo resumido ambas teorías en los capítulos anteriores, en el capítulo \ref{cap:3} se presenta en detalle la analogía propuesta en \cite{Costa-Herdeiro}, para luego hacer uso de las ecuaciones de desvío obtenidas anteriormente para definir los tensores de marea electromagnéticos y gravitacionales, los cuales permitirán comparar ambas teorías. Luego, se hace una primera comparación estudiando campos gravitacionales débiles haciendo uso de teoría de perturbaciones. Y para finalizar, se estudian extensiones a la correspondencia existente para el caso dipolar en las ecuaciones de movimiento en ambos contextos, para luego realizar una discusión sobre las similitudes y diferencias entre la teoría de Relatividad General y la teoría electromagnética clásica.

Por último, en el capítulo \ref{cap:4} se estudia un caso particular en donde al considerar una 4-velocidad de la forma $u^{\mu}=\left[ u^t, 0, 0, u^{\phi} \right]$, se simplifican las ecuaciones de movimiento al punto de que es posible obtener soluciones analíticas. Este análisis surge por estudiar el análogo gravitacional de un problema presentado en \cite{Costa-Natario-Zilhao}, donde se puede hacer una simplificación similar considerando la misma 4-velocidad.