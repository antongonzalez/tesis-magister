\chapter*{Introducci\'on}

En \cite{Costa-Herdeiro} se plantea una nueva forma de establecer una relación entre la teoría de Relatividad General y la teoría electromagnética clásica. Los tensores de marea, que permiten el estudio de las inhomogeneidades de los campos en cuestión, al ser usados para escribir las ecuaciones de movimiento en ambas teorías permiten dar cuenta de una completa correspondencia entre los elementos que describen el movimiento de un dipolo magnético y una partícula con espín \cite{Steinhoff-Puetzfeld}.

Para poder entender lo anterior de una forma más clara, en el capítulo \ref{cap:1} se introducen temas relacionados con la teoría electromagnética clásica \cite{gr2}, desde las ecuaciones de Maxwell (y su respectiva formulación covariante), la obtención de la ecuación de desvío para un par de cargas de prueba, finalizando con la expansión multipolar \cite{Dixon1,Dixon2,Dixon3} y el estudio particular para un dipolo magnético.

Luego, en el capítulo \ref{cap:2} se introducen los distintos elementos de Relatividad General \cite{gr1,inverno,Steven,Hans,Poisson}, desde presentar las ecuaciones de campo de Einstein, explicando el principio de equivalencia y describiendo la solución de Kerr \cite{Heinicke}, para después obtener la ecuación de desvío geodésico, y finalizar con la expansión multipolar gravitacional hasta orden cuadrupolar.

Resumiendo ambas teorías en los capítulos anteriores, en el capítulo \ref{cap:3} se presenta en detalle la analogía propuesta en \cite{Costa-Herdeiro}, para luego hacer uso de las ecuaciones de desvío obtenidas anteriormente para definir los tensores de marea electromagnéticos y gravitacionales. Después se hace una primera comparación para el caso de campos gravitacionales débiles, siguiendo con estudiar extensiones a órdenes superiores para la analogía existente en el caso dipolar presentado en \cite{Costa-Herdeiro}. Luego, se finaliza el capítulo con una discusión sobre las principales similitudes y diferencias entre la teoría de Relatividad General y la teoría electromagnética clásica que quedan en evidencia al estudiar esta analogía.

Por último, en el capítulo \ref{cap:4} se resuelven las ecuaciones de Mathisson-Papapetrou para órbitas circulares sobre el plano ecuatorial en la métrica de Kerr. Notamos la existencia de diferentes zonas en el espacio de velocidad angulares donde el 4-momentum es un vector tipo-tiempo o tipo-espacio. Este comportamiento de la solución nos lleva a pensar que existen diferentes zonas donde, para ciertas velocidades angulares, es necesario considerar un mayor número de órdenes en la expansión multipolar para una correcta descripción de la dinámica del giróscopo, o bien considerar restricciones adicionales sobre las variables, por ejemplo el espín. Más adelante, casos particulares de esta solución son estudiados con el fin de comparar nuestros resultados con otros trabajos relacionados conocidos, tales como los presentados en \cite{Costa-Herdeiro-Natario-Zilhao} donde se resuelve un problema similar en un espacio tiempo de Minkowski, y \cite{Costa-Natario-Zilhao} donde se mencionan geodésicas circulares de un giróscopo alrededor de un agujero negro de Kerr.

Un artículo académico fue enviado a la revista \textit{``International Journal of Modern Physics D''} como fin del trabajo realizado durante mi estadía en el programa de postgrado de la Universidad de Concepción.